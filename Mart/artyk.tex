\documentclass[a4paper]{article}
\usepackage[left=1.5cm, right=2.5com, top=2.5, bottom=1.5]{geometry}
\usepackage[MeX]{polski}
\usepackage[utf8]{inputenc}
\usepackage{graphicx}
\usepackage{enumerate}
\usepackage{amsmath}
\usepackage{amssymb}
\title{Przepisy kulinarne}
\author{Martin Kosiński}
\begin{center}


\begin{document}

\section{\huge\textbf{Witam serdecznie na moim  Artykule!}}

\subsection{ \huge O mnie }
\baselineskip 20 pt 
\paragraph{Nazywam się Martin Kosiński moja kulinarna przygoda zaczęła się dość  nie dawno. Od zawsze  pasjonowałem się kuchnia  i gotowaniem lecz  nie miałem na  to czasu  lecz stwierdziłem że podzielę  się swoimi  pomysłami  na  potrawy  gdy  ma  ktoś  nie wiele  czasu  na gotowanie a chcę dobrze  i  smacznie zjeść. }

\maketitle



\subsubsection{\huge\textit{Zapiekane ziemniaczki z sosem czosnkowym}}


\Large \textbf {Składniki:}

\begin{enumerate}[*]
\raggedleft
\item 8 dużych ziemniaków\newline
\item 1 płaska łyżka soli\newline
\item łyżka pełna przyprawy do ziemniaków\newline
\item 3 łyżki oleju\newline
 
\item 125ml jogurtu naturalnego\newline
\item 4 łyżki majonezu\newline
\itam 3 ząbki czosnku\newline
\item 1 łyżeczka przyprawy w płynie maggi\newline
\item szczypta pieprzu\newline
\end{enumerate}
\Large \textbf{Sposób przygotowania:}

Zapiekane ziemniaczki z sosem czosnkowym

\raggedright
Ziemniaki obieramy i kroimy na ósemki. Zalewamy wodą, solimy i stawiamy na ogień.\newline Po ok. \textbf{3 minutach} od momentu zagotowania odcedzamy i odparowujemy. \\Do garnka wlewamy olej i przyprawę do ziemniaków. \\Dokładnie mieszamy i przesypujemy na blachę wyłożoną papierem do pieczenia bądź sylikonową matą. Wkładamy do rozgrzanego do \underline{200°C piekarnika i smażymy ok 20 minut na grzałce góra-dół}, w międzyczasie raz mieszając.\\ Następnie włączamy opiekanie lub grill i pieczemy jeszcze 10 minut.\\ W tym czasie z jogurtu, majonezu, przeciśniętego przez praskę czosnku oraz magi i pieprzu przygotowujemy sos czosnkowy. Podajemy na ciepło. 
\end{center}
\end{document}
